\documentclass{article}
\usepackage{amsmath,amsthm,amssymb}
\usepackage{mathtext}
\usepackage[T1,T2A]{fontenc}
\usepackage[utf8]{inputenc}
\usepackage[english,russian]{babel}
\usepackage[unicode, pdftex]{hyperref}

\newcommand{\DEF}[1] {
    \textbf{Определение: } #1
}

\newcommand{\THRM}[1] {
    \textbf{Теорема: } #1
}

\newcommand{\PROOF}[1] {
    \textbf{Доказательство: } $\square$ #1 $\blacksquare$
}

\newcommand{\eps}{\varepsilon}

\newcommand{\EXMP} {\textbf{Примеры: }}

\DeclareMathOperator{\rank}{rank}

\begin{document}
    \section{Общие слова о линейных кодах}

    Пусть задано продмножество $\mathfrak{K} \subset V_n$ множества двоичнх векторов длины $n$, называемое кодом и пусть задана матрица $H$.

    \DEF{$(n, k)$-кодом называется код со словами длиной $n$ в каждом из которых содержится $k$ информационных символов.}

    \DEF{Линейным называется код, каждый вектор которого удовлетворяет уравнению $Hc^t = 0$. Матрица $H$ называется проверочной матрицей 
    кода $\mathfrak{K}$.}

    \DEF{Код называется групповым, если множество его слов образует группу.}

    Очевидно, что каждый линейный код является групповым (более того, множество его слов образует продпространоство в пространстве $V_n$).

    \EXMP 
    \begin{enumerate}
        \item Рассмотрим код с повторением. $\mathfrak{K} = \{ (c_1, c_2, \dots , c_n) | c_1 = c_i, i=1\dots n \}$
            Его проверочной матрицей будет 
            $$
            H = \begin{pmatrix}
                1 & 1 & 0 & \dots & 0 \\
                1 & 0 & 1 & \dots & 0 \\
                \vdots & \vdots & \vdots & \ddots & \vdots\\
                1 & 0 & 0 & \dots & 1 
            \end{pmatrix}
            $$
        \item Код с проверкой на четность. $\mathfrak{K} = \{ (c_1, c_2, \dots , c_n) | c_n = \sum_{i = 1}^{n - 1} c_i \}$
            Проверочная матрица:
            $$
            H = (1, 1, \dots , 1)
            $$
    \end{enumerate}

    \DEF{Будем говорить что проверочная матрица записана в каноническом виде если $H = (A | -I)$, где $I$ -- единичная матрица}

    В случае когда проерочная матрица записана в каноническом виде очень просто отделить информационные и и проверочные символы:
    первые $k$ -- информационные, остальные -- проверочные.

    Пусть $u = (\alpha_1, \alpha_2, \dots ,\alpha_k)$ -- исхоное информационное слово. Тогда 
    $$
        c^t = \begin{pmatrix} I \\ -- \\ -A \end{pmatrix} u^t
    $$
    Транспонировав равенство получим $c = uG$, где $G = (I | -A^t)$. Матрица $G$ называется порождающей матрицей. Из полученного результата сразу вытекает следующий результат
    Матрицы $G$ и $H$ связаны соотношением $HG^t = 0$.

    На самом деле матрица $G$ иметь такой вид не обязана. Это справедливо для кодов с матрицей $H$ в \underline{ каноническом виде}, то есть для \\ 
    \underline{систематических кодов}.
    
    Есть более общий способ получить матрицу $G$. 

    Посмотрим на равенство $Hc^t = 0$ как на однородную систему линейных уравнений. Посмотрим базис решений этой системы, пусть это будут $\{e_1, e_2, \dots , e_k\}$.
    Тогда любой вектор $c$ мы сможем записать как линейную комбинацию этих векторов $c = \sum_{i = 1}^k \alpha_i e_i$, где $\alpha_i \in \mathbb{Z}_2$. Далее запишем
    найденные решения в строки матрицы $G$. Таким образом получим порождающую матрицу.

    Заметим, что раз строки матрицы $G$ есть ничто иное как решение уравнения $Hc^t = 0$, то получаем следующую теорему:

    \THRM{Матрицы $H$ и $G$ связаны соотношением $HG^t = 0$.}

    \section{Как определять и корректировать ошибки}

    Перед тем как идти дальше следует сделать несклько замечаний. Во-первых, имея информационное слово длины $k$ мы не можем просто так выбрать кодирующую матрицу $G$. 
    
    Чтобы декодировать принятое сообщение $c$, формально, нам нужно найти его прообраз, а это ничто иное как решение системы уравнений $uG = c$. 
    Это значит что $\rank G = k$. Аналогично, $\rank H = n - k$

    \THRM{Пусть код $\mathfrak{K}$ имеет минимальное расстояние $\ge d$ $\Leftrightarrow$ любые $d - 1$ столбцов матрицы $H$ линейно независимы.}

    \PROOF{
        $\Rightarrow$ Заметим что минимальное расстояние линейного кода равно минимальному весу его ненулевого слова.

        Пусть $w(\mathfrak{K}) = d$, и пусть $c = (\alpha_1, \alpha_2, \dots , \alpha_n) \in \mathfrak{K}$ -- кодовое слово 
        и $H = (h_1, h_2, \dots , h_n)$ -- проверочная матрица кода ($h_i$ -- столбцы $H$). Из соотношения $Hc^t = 0$ имеем
        $\alpha_1 h_1 + \alpha_2 h_2 + \dots + \alpha_n h_n = 0$. Пусть $w(c) = d$. Тогда существует ровно $d$ линейно зависимых
        столбцов проверочной матрицы.

        Пусть теперь известно что $t$ столбцов матрицы $H$ линейнозависимы, то есть справедливо равенство
        $$ \eps_{i_1} h_{i_1} + \eps_{i_2} h_{i_2} + \dots \eps_{i_t} h_{i_t} = 0. $$
        Это значит что существует вектор $x$ такой что на позициях $i_j, j = 1\dots t$ у него стоят 1, а на других 0.

        Значит для такого вектора $x$ справедливо равенство $Hx^t = 0$, тогда, с учетом того что $w(\mathfrak{K}) = d$, получаем $t \ge d$.
        Следовательно любые $d - 1$ столбцов матрицы $H$ линейно независимы.

        $\Leftarrow$ Обратно, пусть любые $d - 1$ столбцов $H$ -- линейнонезависимы, тогда если $c \in \mathfrak{K}$, то $w(c) \ge d$.
    }

    Будем считать что шум в канале просто прибавлет к нашему слову $c$ вектор ошибки $e$. Пусть полученное слово будет $y = c + e$. Тогда 
    справедлива

    \THRM{Групповой код $\mathfrak{K}$ оставляет незамеченными те и только те ошибки, которые являются его элементами.}

    Предположим что при передаче ошибки происходят независимо друг от друга с вероятностью $q < \frac{1}{2}$. Тогда, воспользовавшись
    законом Бернулли, вероятность того что проихошло ровно $w$ ошибок равна $C_n^w q^w(1 - q)^{n - w}$. Отсюда видно что при $q < 1/2$
    вероятность появления шумового слова веса $t$ меньше, чем вероятность появления шумового слова веса $t - 1$. 

    Заметим что если происходит ошибка, то вектор ошибки будет находится в смещном классе $\mathfrak{K} + y$. Обратно, если $c' + g = y$, 
    то $\mathfrak{K} + g = \mathfrak{K} + y$. Поэтому множество векторов ошибок в точности составлет смежный класс $\mathfrak{K} + y$.

    Поэтому возникает следующаяя идея: выпишем все возможные элементы смежного класса $\mathfrak{K} + y$ и среди них выберем слово
    наименьшего веса, которое назовем лидером смежного класса. Справедлива

    \THRM{Групповой код $\mathfrak{K}$ исправлет в точности те ошибки, которые являются лидерами смежных классов.}

    Сформулируем еще одну теорему о линейных кодах

    \THRM{Линейны код исправлет одиночные ошибки тогда и только тогда когда все столбцы матрицы $H$ отличны от нуля и различны.}

    \PROOF{
        $\Rightarrow$ Пусть $H = (h_1, h_2, \dots , h_n)$ и $e_i = (0, 0, \dots , 0, 1, 0, \dots , 0)$, где 1 стоит на $i$ позиции.
        Тогда $H(c + e_i) = Hc + He_i = h_i$. Так как нам нужно чтобы мы умели различать ошибку в позиции $i \neq j$, то столбец $h_i \neq h_j$ и
        $h_i \neq 0$ при всех $i = 1..n$. 

        $\Leftarrow$ Так как все столбцы матрицы $H$ различны и отличны от нуля, то $H(c + e_i) = He_i = h_i$, то можно построить следующее 
        соответствие 
        $$
        \begin{matrix}
            h_1 \rightarrow 1 \\
            h_2 \rightarrow 2 \\
            \vdots \\
            h_n \rightarrow n
        \end{matrix}
        $$

        То есть если мы получили слово $y$ и $Hy = h_i$ это значит что при передаче произошла ошибка в позиции $i$. 
    }

    Результат применения матрицы $H$ к полученному слову $y$ называется синдромом ошибки.

    \THRM{
        Пусть $H$ -- проверочная матрица кода $\mathfrak{K}$, тогда синдром ошибки $S$ есть сумма столбцов соответствующих тем позициям в 
        которых произошли ошибки. Два вектора $x$ и $y$ имеет одинаковый синдром, если они находятся в одном смежном классе.
    }

    Из теоремы следует что можество всех синдромов образует факторгруппу $\mathbb{Z}_2^n / \mathfrak{K}$.

    \EXMP{
        Построим код $\mathfrak{K}$ наибольшей длины $r$ исправляющий ровно одну ошибку. Ненулевых двоичных слов длины $r$ сущетсвует
        $2^r - 1$ штук. Разместим их матрице следующим образом (для $r = 3$): 
        $$
        \begin{pmatrix}
            0 & 0 & 0 & 1 & 1 & 1 & 1 \\  
            0 & 1 & 1 & 0 & 0 & 1 & 1 \\
            1 & 0 & 1 & 0 & 1 & 0 & 1
        \end{pmatrix}
        $$

        В столбце с номером $k$ стоит двоичная запись числа $k$. Тогда если мы прибавим вектор ошибки $e_k$, то соответствующий синдром
        будет представлять двоичную запись числа $k$, а значит, переведя его в десятичную, мы получим номер позиции в которой произошла 
        ошибка.
    }

\end{document}